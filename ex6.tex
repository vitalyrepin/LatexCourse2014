\documentclass[a4paper]{book}
\usepackage[utf8]{inputenc}
\usepackage[T1,T2A]{fontenc}
\usepackage[english,russian]{babel}
\usepackage{cmap}
\usepackage{hyperref}
\usepackage{indentfirst}
\usepackage{mathabx}
\usepackage{xcolor}
\usepackage{fancyhdr}
\usepackage{enumitem}
\usepackage{calc}

\frenchspacing

\renewcommand{\headrulewidth}{0mm}
\chead{}\rhead{}\lhead{}
\cfoot{\sc\thepage}\rfoot{}\lfoot{}
\pagestyle{fancy}

\hypersetup{%
unicode=true,           % русские буквы в раздела PDF
pdflang={ru},
colorlinks=true,       	% false: ссылки в рамках; true: цветные ссылки
linkcolor=cyan,          % внутренние ссылки
citecolor=green,        % на библиографию
filecolor=magenta,      % на файлы
urlcolor=blue           % на URL
}

\begin{document}
\pagenumbering{Roman}
\author{Сергей Сергеев}
\title{Magnum opus. Номера страниц "--- внизу}
\maketitle
\section*{Предисловие}

Первое издание этой книги вышло десять лет назад. За это время некоторые вещи, изложенные в первом издании, устарели, что потребовало внесения поправок в текст. И вот, спустя год плодотворной работы, я рад наконец представить на суд читателей \textbf{\textit{второе издание}} этой книги, исправленное и дополненное.

\subsection*{Изменения по сравнению с первым изданием}

\begin{itemize}[label=\hspace*{.3cm}\textcolor{blue}{$\sqbullet$},align=left,labelwidth=.5em,itemindent=\labelwidth+\labelsep,leftmargin=0.3cm]
\item Первые четыре главы книги переписаны с учетом изменений, произошедших за последнее десятилетие.
\item Добавлено несколько новых глав.
\item Дополнительные материалы ко всем главам книги доступны теперь на сайте \url{http://www.kakayatokniga.com}.
\end{itemize}

\subsection*{Благодарность}

Благодарю за ценные советы и комментарии \emph{Петра Петрова} из Какого-то университета и \emph{Ивана Иванова} из Какого-то другого университета.
\medskip

\rightline{Сергей \textsc{Сергеев}}

\mainmatter
\chapter{Иммануил Кант.  ИДЕЯ ВСЕОБЩЕЙ ИСТОРИИ ВО ВСЕМИРНО-ГРАЖДАНСКОМ ПЛАНЕ}

Какое бы точное понятие мы ни составили себе с метафизической точки зрения о свободе воли, необходимо, однако, признать, что проявления воли, человеческие поступки, подобно всякому другому явлению природы определяются общими законами природы. История, занимающаяся изучением этих проявлений, как бы глубоко ни были скрыты их причины, позволяет думать, что если бы она рассматривала игру свободы человеческой воли в совокупности, то могла бы открыть ее закономерный ход; и то, что представляется запутанным и не поддающимся правилу у отдельных людей, можно было бы признать по отношению ко всему роду человеческому как неизменно поступательное, хотя и медленное, развитие его первичных задатков. Так, браки, обусловливаемые ими рождения и смерти, на которые свободная воли человека имеет столь большое влияние, кажутся не подчиненными никакому правилу, на основании которого можно было бы наперед математически определить их число. Между тем ежегодные данные о них в больших странах показывают, что они так же происходят согласно постоянным законам природы, как те столь изменчивые колебания погоды, которые в единичных случаях нельзя заранее определить, но которые в общем непрерывно и равномерно поддерживают произрастание злаков, течение рек и другие устроения природы. Отдельные люди и даже целые народы мало думают о том, что когда они, каждый по .своему разумению и часто в ущерб другим, преследуют свои собственные цели, то они незаметно для самих себя идут к неведомой им цели природы как за путеводной нитью и содействуют достижению этой цели, которой, даже если бы она стала им известна, они бы мало интересовались.

Так как люди в своих стремлениях действуют в общем не чисто инстинктивно, как животные, но и не как разумные граждане мира, по согласованному плану, то кажется, что и не может быть у них планомерной истории (так же как, скажем, у пчел или бобров). Нельзя отделаться от некоторого неудовольствия, когда видишь их образ действий на великой мировой арене. Тогда находишь, что при всей мнимой мудрости, кое-где обнаруживающейся в частностях, в конечном счете все в целом соткано из глупости, ребяческого тщеславия, а нередко и из ребяческой злобы и страсти к разрушению. И б конце концов не знаешь, какое себе составить понятие о нашем роде, столь убежденном в своих преимуществах. Для философа здесь остается один выход: поскольку нельзя предполагать у людей и в совокупности их поступков какую-нибудь разумную собственную цель, нужно попытаться открыть в этом бессмысленном ходе человеческих дел цель природы, на основании которой у существ, действующих без собственного плана, все же была бы возможна история согласно определенному плану природы. — Посмотрим, удастся ли нам найти путеводную нить для такой истории, и тогда предоставим природе произвести того человека, который был бы в состоянии ее написать. Ведь породила же она Кеплера2, подчинившего неожиданным образом эксцентрические орбиты планет определенным законам, и Ньютона3, объяснившего эти законы общей естественной причиной.


\section{Положение первое}

Все природные задатки живого существа предназначены для совершенного и целесообразного развития. Это подтверждают внешнее наблюдение над всеми животными и изучение их анатомии. Орган, не имеющий применения, устройство, не достигающее своей цели, представляют собой противоречие в телеологическом учении о природе. В самом деле, если мы отказываемся от этих основоположений, то имеем не закономерную, а бесцельно играющую природу; и, как ни печально, вместо разума путеводной нитью становится случай.

\section{Положение второе}

Природные задатки человека (как единственного разумного существа на земле), направленные на применение его разума, развиваются полностью не в индивиде, а вроде. Разум, которым наделено существо, — это способность расширять за пределы природного инстинкта правила и цели приложения всех его сил: замыслам его нет границ. Но сам разум не действует инстинктивно, а нуждается в испытании, упражнении и обучении, дабы постепенно продвигаться от одной ступени проницательности к другой. Вот почему каждому человеку нужно непомерно долго жить, чтобы научиться наиболее полно использовать свои природные задатки; или если природа установила лишь краткий срок для его существования (как это и есть на самом деле), то ей нужен, быть может, необозримый ряд поколений, которые последовательно передавали бы друг другу свое просвещение, дабы наконец довести задатки в нашем роде до такой степени развития, которая полностью соответствует цели. И этот момент должен быть, по крайней мере в мыслях человека, целью его стремлений, иначе природные задатки следовало бы рассматривать большей частью как бесполезные и бесцельные; а это свело бы на нет все практические принципы и позволило бы заподозрить природу, мудрость которой должна служить правилом при рассмотрении всех прочих установлений, в том, что только с человеком она сыграла глупую шутку.

\section{Положение третье}

Природа хотела, чтобы человек все то, что находится за пределами механического устройства его животного существования, всецело произвел из себя и заслужил только то счастье или совершенство, которое он сам создает свободно от инстинкта, своим собственным разумом. Природа не делает ничего лишнего и не расточительна в применении средств для своих целей. Так, как она дала человеку разум и основывающуюся на нем свободную волю, то уже это было ясным свидетельством ее намерения наделить его [способностями], Она не хотела, чтобы он руководствовался инстинктом или был обеспечен прирожденными знаниями и обучен им, она хотела, чтобы он все произвел из себя. Изыскание средств питания, одежды и крова, обеспечение внешней безопасности и защиты (для чего она дала ему не рога быка, не когти льва и не зубы собаки, а только руки), все развлечения, могущие сделать жизнь приятной, даже его проницательность и ум, даже доброта его воли — все это должно быть исключительно делом его рук. Природа, кажется, здесь сама находит удовольствие в величайшей бережливости, и она так скупо наделила людей животными качествами, так строго нацелила уже первоначальное существование их на высшую потребность, как если бы она хотела, чтобы человек, когда он от величайшей грубости возвысится до величайшей искусности, до внутреннего совершенства образа мыслей (поскольку это возможно на земле) и благодаря этому достигнет счастья, — чтобы только он воспользовался плодами своих трудов и был обязан ими только самому себе. Похоже на то, что она рассчитывала больше на его разумную самооценку, чем на его внешнее благополучие. Ведь на этом пути человека ждут неисчислимые трудности. Кажется, однако, что природа беспокоилась вовсе не о том, чтобы человек жил хорошо, а о том, чтобы он сам достиг такого положения, когда благодаря своему поведению он станет достойным жизни и благополучия. При этом всегда удивляет то, что старшие поколения трудятся в поте лица как будто исключительно ради будущих поколений, а именно для того, чтобы подготовить им ступень, на которой можно было бы выше возводить здание, предначертанное природой, и чтобы только позднейшие поколения имели счастье жить в этом здании, для построения которого работал длинный ряд предшественников (хотя, конечно, не преднамеренно), лишенных возможности пользоваться подготовленным ими счастьем. Но каким бы загадочным ни казался такой порядок, он необходим, если раз навсегда признать, что одаренные разумом животные, которые, как класс разумных существ, все смертны, но род которых бессмертен, должны достигнуть полного развития своих задатков.

\section{Положение четвертое}

Средство, которым природа пользуется для того, чтобы осуществить развитие всех задатков людей, — это антагонизм их в обществе, поскольку он в конце концов становится причиной их законосообразного порядка. Под антагонизмом я разумею здесь недоброжелательную общительность людей, т.е. их склонность вступать в общение, связанную, однако, с всеобщим сопротивлением, которое постоянно угрожает обществу разъединением. Задатки этого явно заложены в человеческой природе. Человек имеет склонность общаться с себе подобными, ибо в таком состоянии он больше чувствует себя человеком, т.е. чувствует развитие своих природных задатков. Но ему также присуще сильное стремление уединяться (изолироваться), ибо он в то же время находит в себе необщительное свойство — желание все сообразовать только со своим разумением — и поэтому ожидает отовсюду сопротивление, так как он по себе знает, что сам склонен сопротивляться другим. Именно это сопротивление пробуждает все силы человека, заставляет его преодолевать природную лень, и, побуждаемый честолюбием, властолюбием или корыстолюбием, он создает себе положение среди своих ближних, которых он, правда, не может терпеть, но без которых он не может и обойтись. Здесь начинаются первые истинные шаги от грубости к культуре, которая, собственно, состоит в общественной ценности человека. Здесь постепенно развиваются все таланты, формируется вкус и благодаря успехам просвещения кладется начало для утверждения образа мыслей, способного со временем превратить грубые природные задатки нравственного различения в определенные практические принципы и тем самым патологически вынужденное согласие к жизни в обществе претворить в конце концов в моральное целое. Без этих самих по себе непривлекательных свойств необщительности, порождающих сопротивление, на которое каждый неизбежно должен натолкнуться в своих корыстолюбивых притязаниях, все таланты в условиях жизни аркадских пастухов, [т.е.] в условиях полного единодушия, умеренности и взаимной любви; навсегда остались бы скрытыми в зародыше; люди, столь же кроткие, как овцы, которых они пасут, вряд ли сделали бы свое существование более достойным, чем существование домашних животных; они не заполнили бы пустоту творения в отношении цели его как разумного естества. Поэтому да будет благословенна природа за неуживчивость, за завистливо соперничающее тщеславие, за ненасытную жажду обладать и господствовать! Без них все превосходные природные задатки человечества оставались бы навсегда неразвитыми. Человек хочет согласия, но природа лучше знает, что для его рода хорошо; и она хочет раздора. Он желает жить беспечно и весело, а природа желает, чтобы он вышел из состояния нерадивости и бездеятелььного довольства и окунулся с головой в работу и испытал трудности, чтобы найти средства разумного избавления от этих трудностей. Таким образом, естественные побудительные причины, источники необщительности и всеобщего сопротивления, вызывающие столько бедствий, но и беспрестанно побуждающие человека к новому напряжению сил и, стало быть, к большему развитию природных задатков, прекрасно обнаруживают устройство, созданное мудрым творцом; и здесь вовсе ни при чем злой дух, который будто бы вмешивается в великолепное устроение, созданное творцом, или из зависти портит его.

\section{Положение пятое}

Величайшая проблема для человеческого рода, разрешить которую его вынуждает природа, — достижение всеобщего правового гражданского общества. Только в обществе, и именно в таком, в котором членам его предоставляется величайшая свобода, а стало быть, существует полный антагонизм и тем не менее самое точное определение и обеспечение свободы ради совместимости ее со свободой других, — только в таком обществе может быть достигнута высшая цель природы: развитие всех ее задатков, заложенных в человечестве; при этом природа желает, чтобы эту цель, как и все другие предначертанные ему цели, оно само осуществило. Вот почему такое общества, в котором максимальная свобода под внешними законами сочетается с непреодолимым принуждением, т.е. совершенно справедливое гражданское устройство, должно быть высшей задачей природы для человеческого рода, ибо только посредством разрешения и исполнения этой задачи природа может достигнуть остальных своих целей в отношении нашего рода. Вступать в это состояние принуждения заставляет людей, вообще-то расположенных к полной свободе, беда, и именно величайшая из бед — та, которую причиняют друг другу сами люди, чьи склонности приводят к тому, что при необузданной свободе они не могут долго ужиться друг с другом. Однако в таком ограниченном пространстве, как гражданский союз, эти же человеческие склонности производят впоследствии самое лучшее действие подобно деревьям в лесу, которые именно потому, что каждое из них старается отнять у другого воздух и солнце, заставляют друг друга искать этих благ все выше и благодаря этому растут красивыми и прямыми; между тем как деревья, растущие на свободе, обособленно друг от друга, выпускают свои ветви как попало и растут уродливыми, корявыми и кривыми. Вся культура и искусство, украшающие человечество, самое лучшее общественное устройство — все это плоды необщительности, которая в силу собственной природы сама заставляет дисциплинировать себя и тем самым посредством вынужденного искусства полностью развить природные задатки.

\section{Положение шестое}

Эта проблема самая трудная и позднее всех решается человеческим родом. Трудность, которую ясно показывает уже сама идея этой задачи, состоит в следующем; человек есть животное, которое, живя среди других членов своего рода, нуждается в господине. Дело в том, что он обязательно злоупотребляет своей свободой в отношении своих ближних; и хотя он, как разумное существо, желает иметь закон, который определил бы границы свободы для всех, но его корыстолюбивая животная склонность побуждает его, где это ему нужно, делать для самого себя исключение. Следовательно, он нуждается в господине, который сломил бы его собственную волю и заставил его подчиняться общепризнанной воле, при которой каждый может пользоваться свободой. Где же он может найти такого господина? Только в человеческом роде. Но этот господин также есть животное, нуждающееся в господине. Поэтому, как ни поступит человек в данном случае: предоставит ли он верховную власть одному или сообществу многих избранных для этой цели лиц, нельзя понять, как он создаст себе главу публичной справедливости, который сам был бы справедлив. Ведь каждый облеченный властью всегда будет злоупотреблять своей свободой, когда над ним нет никого, кто распоряжался бы им в соответствии с законами. Верховный глава сам должен быть справедливым и в то же время должен быть человеком. Вот почему эта задача самая трудная из всех; более того, полностью решить ее невозможно; из столь кривой тесины, как та, из которой сделан человек, нельзя сделать ничего прямого. Только приближение к этой идее вверила нам природа*. Что эта проблема решается позднее всех, следует еще из того, что для этого требуются правильное понятие о природе возможного [государственного] устройства, большой, в течение многих веков приобретенный опыт и, сверх того, добрая воля, готовая принять такое устройство. А сочетание этих трех элементов — дело чрезвычайно трудное, и если оно будет иметь место, то лишь очень поздно, после многих тщетных попыток.

\end{document}
