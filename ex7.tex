\documentclass[a4paper]{article}
\usepackage[utf8]{inputenc}
\usepackage[T1,T2A]{fontenc}
\usepackage[english,russian]{babel}
\usepackage{amsthm}

\theoremstyle{definition}

\newtheorem{case}{Случай}

\usepackage{geometry} % Простой способ задавать поля (из шаблонов курса)
\geometry{top=25mm}
\geometry{bottom=35mm}
\geometry{left=35mm}
\geometry{right=20mm}

%% Resetting case counter when section counter is changed
\makeatletter
\@addtoreset{case}{section}
\makeatother

\begin{document}

\section{Три девицы под окном}
\begin{case} Кабы я была царица, "---
Говорит одна девица, "---
То на весь крещеный\\ мир
Приготовила б я пир.
\end{case}

\begin{case}
Кабы я была царица, "---
Говорит ее сестрица, "---
То на весь бы мир\\ одна
Наткала я полотна.
\end{case}

\begin{case}
Кабы я была царица, "---
Третья молвила сестрица, "---
Я б для батюшки-\\ царя
Родила богатыря.
\end{case}

\section{Камень}
\begin{case}
Направо пойдешь "--- жену найдёшь
\end{case}

\begin{case}
Налево "--- коня потеряешь
\end{case}

\begin{case}
Прямо "--- сам пропадёшь
\end{case}
\end{document}
