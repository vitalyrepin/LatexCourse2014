\documentclass[a4paper]{book}
\usepackage[utf8]{inputenc}
\usepackage[T1,T2A]{fontenc}
\usepackage[english,russian]{babel}
\usepackage{cmap}
\usepackage{tabularx}
\usepackage{geometry} % Простой способ задавать поля

\pagestyle{empty}

\frenchspacing

%% From course templates
\geometry{top=25mm}
\geometry{bottom=35mm}
\geometry{left=35mm}
\geometry{right=20mm}

\begin{document}

%% OK, not exactly the same as in the example. However, I hope it is close enough.
\begin{tabularx}{\textwidth}{X|p{3cm}|X}
\bf \rightline{\bf Сертификат} \rightline{\bf об окончании} & \bf Сертификат об\newline окончании с отличием & \bf Подтвержденный\newline сертификат\\
\hline
\hline
В течение курса студентам будут предложены 4 теста и 1 финальное взаимнооцениваемое задание \emph{(peer assessment).}\newline  Каждый тест составляет 25~\% от итоговой оценки, при этом в формулу ее расчета идут только 3 лучшие теста из 4.\newline Таким образом, тесты в сумме составляют 75~\% от итоговой оценки.\newline Взаимнооцениваемое задание составляет
25~\% от итоговой оценки.\newline Для получения сертификата об окончании необходимо набрать не менее \textbf{60~\%} от максимально возможного балла.& 
Для получения необходимо набрать не менее \textbf{80~\%} от максимально возможного балла.&
Система оценивания та же.\newline Для получения подтвержденного сертификата необходимо набрать не менее \textbf{60~\%} от максимально возможного балла, для получения подтвержденного сертификата с отличием \emph{(Verified Certificate with Distinction)} необходимо набрать не менее \textbf{80~\%} от максимально возможного балла.
\end{tabularx}
\end{document}
